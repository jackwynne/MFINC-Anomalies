\documentclass[a4paper,12pt]{article}                 % Tell LaTeX to use the "article" style
\usepackage[paper=a4paper,dvips,top=1.5cm,left=1.5cm,right=1.5cm,foot=1cm,bottom=1.5cm]{geometry}
\usepackage{import}
\usepackage{placeins}
\usepackage{booktabs}
\usepackage{multirow}
\usepackage{graphicx}
\usepackage{natbib}
\usepackage[labelfont=bf,labelsep=none]{caption}
\usepackage{titlesec}
\newcommand{\sectionbreak}{\clearpage}
\usepackage{breqn}
\setlength{\parskip}{1em}

\begin{document}                        % Document starts here
\title{Revisiting the Anomalies }
\author{Jack Wynne}
\date{January 2020}
\maketitle
\section{Introduction}
\section{Literature Review}
\subsection{Stocks with Extreme Past Returns: Lotteries or Insurance?}
In his 2018 paper, Barinov investigates the relationship between stocks with lottery-like returns and market volatility. A stock that offers a small chance of having a vast payoff is said to be lottery-like. Lottery-like stocks also tend to be growth stocks and have high idiosyncratic volatility. In his earlier work, Barinov has shown that growth stocks with high idiosyncratic volatility act as hedges against aggregate volatility risk and therefore, lottery-like stocks should also act as hedges against aggregate volatility risk. The ability for growth and lottery stocks to act as hedges against aggregate volatility risk comes from the way that growth options react positively to increases in volatility and that growth options will suffer a smaller price drop, due to a more modest increase in the options risk and discount rate, when market volatility and risk premiums increase. 

Barinov’s empirical analysis starts by showing that firms that have had previous extreme returns and firms with significant expected idiosyncratic skewness have the conditions required for the aggregate volatility explanation, which are high idiosyncratic volatility and firm-specific uncertainty. Next, Barinov creates an aggregate volatility risk factor (FVIX) based on the daily changes in the VIX index. The VIX index is the implied volatility of the options on the S\&P 500 index and is generally used as a measure of expected aggregate volatility. 

The paper hypothesises that convexity in the firm value is how the aggregate volatility risk explanation of the low returns for lottery-like stocks works. Barinov predicts that the skewness and maximum effects will be higher for growth stocks and that the reason behind this is aggregate volatility risk. In testing, the author finds that the maximum and skewness effects are roughly 60bps per month higher for growth stocks than value stocks. Adding the control of the volatility risk factor (FVIX) the difference falls to 20bps per month, irrespective of which other of the common models are added as controls. Also, the growth like lottery stocks loads more on the FVIX factor than on the value lottery-like stocks. 

\subsection{The Short of It: Investor Sentiment and Anomalies}
Stambaugh, Yu and Yuan explore how investor sentiment affects stock pricing, through a market-wide component with the potential to influence the prices of many securities in the same direction at the same time and the impediment to short selling. This paper looks at whether investor sentiment provides a partial explanation of the asset pricing anomalies that persist after taking the three Fama and French (1993) factors into account.

Following high sentiment, the anomalies should be stronger is the first of three hypotheses that the paper investigates. The second is that when sentiment is high, the returns on the short leg of the portfolios for each anomaly should be lower. Finally, investor sentiment should not affect the long leg of each anomaly portfolio. 

The paper finds that each anomaly is stronger following high investor sentiment levels. The strength of the relationship is shown in an average of 70% of benchmark adjusted profits from a long-short strategy occur in the following months where investor sent investor sentiment was above the median levels and in time series regressions which show a significant positive relationship between investor sentiment and the long-short anomaly portfolio. 

Regarding the second hypothesis, the paper finds that the return in the short leg of the portfolio is lower following high sentiment months. Time series regression shows a significant negative relationship between investor sentiment and the returns on the short leg of the portfolio.

Finally, the paper finds that sentiment does not affect the long leg returns. None of the eleven portfolios has a significant difference in the long legs returns between high and low sentiment periods. In a time-series regression, it is confirmed that there is no relationship between benchmark adjusted long leg returns and the sentiment of investors.

\subsection{The Causal Effect of Limits to Arbitrage on Asset Pricing Anomalies}
In this paper by Chu, Hirshleifer and Ma, they investigate if the return anomalies exist, concerning mispricing, because limits to arbitrage prevent sophisticated investors from profitably trading against them. As a result of the difficulty to purely measure the variations in the limits to arbitrage while excluding changes in other forces that may have an impact on the risk premium or mispricing, the paper studies the causal effects of restrictions to arbitrage on the eleven common pricing anomalies. One of the critical limits to arbitrage is short-sale constraints. The paper uses a pilot program run by the Securities and Exchange Commission on a subset of the stocks on the NYSE, AMEX and NASDAQ that removed short sale price tests that made it easier to short-sell stocks that were in the pilot group. 

Chu, Hirshlefier and Ma examine two hypotheses concerning the differing performance of firms in the pilot versus firms not in the pilot. The first hypothesis is that for the firms in the pilot, the effect of the pricing anomalies should be weaker relative to firms not in the pilot. To test the first hypothesis, the authors split the pilot and nonpilot firms and construct long-short portfolios. In five of the eleven anomalies, the effect of the anomaly is significantly weaker.

The second hypothesis is that the short leg of the portfolios is responsible for the decrease in anomaly returns for pilot stocks during the pilot period. Using the same long-short framework, the short leg portfolios were significantly less profitable. In contrast to the decrease in short leg profitability, the profitability of the long leg did not see any significant effects.

\subsection{Additional Literature Relating to the Pricing Anomalies}
This subsection provides an outline of each of the papers which provided the initial insights into the anomalies that are studied.

Jegadeesh and Titman (1993) look at a strategy that buys stocks that have strong past performance and shorts stocks that have had poor recent performance. Using this momentum strategy, the authors generate significant positive returns with holding periods ranging from 3 months to a year. The returns that are generated in the 12 months after the portfolio is formed disappear in the 24 months that follow. Additionally, the authors note that the profitability of this strategy is not due to systematic risk or a result of a delay in stock price reactions to common factors. 

Novy-Marx (2013) finds that gross profit to assets has approximately the same power as the book-to-market ratio in predicting the cross-section on average returns. Despite having significantly higher valuations, profitable firms generate substantially higher returns than their unprofitable counterparts. The author also finds that controlling for gross profitability also drastically increases the returns from a value strategy and that the effect is especially potent for large firms who have highly liquid stocks. 
Cooper, Gulen and Schill (2008) describe how firm-level asset investment generates subsequent stock returns by examining the relationship between asset growth and returns. The authors find that the growth of a firm's assets is a strong predictor of future abnormal returns and that this effect is not dependent on the size of the firm. 

Titman, Wei and Xie (2004) illustrate a negative relationship between an increase in capital investment and the subsequent benchmark adjusted returns. The negative relationship is stronger for firms where debt ratios are low, and cash flows are high, which give the firm more considerable discretion over its investments.

Hirshleifer, Hou, Teoh, and Zhang (2004) demonstrate that when a firm’s cumulative net operating income is growing faster than its free cash flow, it will see weak earnings growth. The net operating assets effect is identified as resulting from attention limited investors focusing on accounting profitability instead of cash profitability. Net operating assets is a measure of the over-optimism created by the reporting of accounting profits. The authors find that the effect is robust to an extensive range of control variables and different testing methodologies. 

Sloan (1996) provides an investigation into whether stock prices reflect current information about the future of a firm’s earnings that are contained in the accrual and cash flow components of financial statements. The persistence of a firm’s profits into the future depends on the relative magnitude of the firm's cash and accrual positions. Stock prices, on the other hand, are fixated on current earnings and fail to reflect, to the fullest extent, the information contained in the accruals and cash positions until those positions are felt in future earnings. 

Pontiff and Woodgate (2008) delve into the issue of share issuance and returns first touched upon by Loughran and Ritter (1995). The authors find a negative relation between stock issuance and long-run returns and a positive relationship between share repurchases and long-run returns. These results were unaffected by seasoned equity offerings. 

Daniel and Titman (2006) explore how the book-to-market effect is a proxy for intangible returns rather than the interpretation of it being an indicator of high future returns for distressed stocks with poor past performance. The authors find that a stocks future returns are unrelated to the firms past accounting performance. However, it is strongly negatively related to the intangible return. The intangible return is the component of past returns that is orthogonal to the firm’s past performance. Finally, Daniel and Titman note that a measure of composite equity issues is related to intangible returns while independently forecasting returns. 

Campbell, Hilscher and Szilagyi (2008) depict the relationship between the determinants of firm failure and the pricing of financially distressed stocks that have a high probability of failure. The authors estimate failure probability using a dynamic logit model based on accounting and market variables. Financially distressed firms deliver lower returns than their un-distressed counterparts. Firms with a high failure probability are found to have higher standard deviations, market betas and loading on value and small-cap risk. 

Dichev (1998) breaks down the relationship between bankruptcy risk and returns. Bankruptcy risk acts as a proxy for firm distress, which previous studies suggest could be behind the size and book-to-market effects. A positive association between bankruptcy risk and returns would indicate that bankruptcy risk is systematic. However, Dichev finds that that firms with a high risk of bankruptcy earn lower returns, meaning that distress is unlikely to account for the size or book-to-market effects. 

\section{Data: Anomalies, Volatility, Economic Policy Uncertainty and Liquidity}
\subsection{Anomalies: Data Gathering and Variable Construction}
This research explores the eleven previously documented anomalies which still exist even after exposure to the three factors set out by~\cite{fama1993common}. As noted by~\cite{stambaugh2012short}, the use of the Fama and French three factor model rather than the capital asset pricing model (CAPM) as a hurdle is important because just using the CAPM would create an inordinately broad set of anomalies to be tested. 

To calculate the anomalies, CRSP data will be obtained from Wharton Research Data Services (WRDS). The dataset will cover the firms on the S\&P 500 index starting in 1974 and running through to 2018. How the construction is undertaken is outlined below and is based on~\cite{chu2017causal}.

\paragraph*{Anomaly One: Momentum}
The calculation of momentum uses the conventional 11 month ranking period \((t-12 \quad to \quad t-2)\) followed by a one month skip $(t-1)$ and then a one month holding period $(t)$. The skipped month is used to avoid short-run reversal effects.

\paragraph*{Anomaly Two: Gross Profitability}
In line with~\cite{novy2013other}, gross profitability will be calculated as total revenue less the cost of goods sold \((REVT_t-COGS_t)\), scaled by the value of total assets $(AT_t)$. 

\paragraph*{Anomaly Three: Asset Growth}
Asset growth is calculated as the change in total assets \((AT_t  - AT_{t-1})\) and is scaled by lagged total assets \((AT_{t-1})\).

\paragraph*{Anomaly Four: Investment to Assets}
The investment to assets anomaly is calculated as the annual change in gross property, plant and equipment and inventories, again this is scaled by lagged total assets \((AT_{t-1})\). 

\paragraph*{Anomaly Five: Return on Assets}
Return on assets is measured as quarterly earnings scaled by total quarterly assets. 

\paragraph*{Anomaly Six: Net Operating Assets}
To estimate the net operating assets, the difference between all operating assets and all operating liabilities is calculated and then scaled by lagged total assets.
\[
NOA= \frac{Operating Asset_t-Operating Liabilities_t}{AT_{t-1}}\]

\paragraph*{Anomaly Seven: Accruals}
The accrual anomaly is calculated as the changes in non-cash working capital less depreciation expenditure and scaled by lagged total assets.
    \begin{dmath*}
        Accruals_t = \frac{(\Delta CA - \Delta Cash) - (\Delta CL -\Delta STD -\Delta TP)-DAE)}{AT_{t-1}}
    \end{dmath*}

\paragraph*{Anomaly Eight: Net Stock Issues}
To measure the anomaly on an annual basis, the change in the natural logarithm of a firm’s adjusted shares over the previous year. 
\[
NSI_t = \ln (Adjusted \ Shares_t ) - \ln ( Adjusted \ Shares_{t-1})
\]

\paragraph*{Anomaly Nine: Composite Equity Issues}
The composite equity measure decreases because of share repurchases, dividends and any other actions that remove cash from the firm. 
\[
CEI_t = \ln (\frac{Me_t}{Me_{t-5}})  - r(t - 5,t)
\]

\paragraph*{Anomaly Ten: Failure Probability }
A dynamic logit model that matches empirically observed default events using both market and accounting information is used as a measure of failure probability. 
    \begin{dmath*}
        FP_t  =  - 9.16 - 20.26 NIMTAAVG_t+ 1.42 TLMTA_t  - 7.13 EXRETAVG_t + 1.41 SIGMA_t  - 0.045 RSIZE_t - 2.13 CASHMTA_t + 0.075 MB_t - 0.058 PRICE_t
    \end{dmath*}

\paragraph*{Anomaly Eleven: O-Score}
Another measure of financial distress is the O-score proposed by\cite{ohlson1980financial}.

    \begin{dmath*}
        OS_t= - 1.32 - 0.407 \log (ADJASSET_t )  + 6.03 TLTA_t- 1.43 WCTA_t+ 0.076 CLCA_t- 1.72O ENEG_t- 2.37 NITA_t- 1.83 FUTL_t+ 0.285 INTWO_t- 0.521 CHIN_t
    \end{dmath*}

\subsubsection{Anomaly Long-Short Portfolios}
For each of the eleven anomalies, all of the stocks are sorted from best to worst and split into decile portfolios. Value weighted returns are calculated for each decile portfolio. A long-short strategy is implemented using the best and worst deciles, going long on the high performing decile and short on the worst performing decile. 

\subsection{Additional Data}
\subsubsection{Fama French Factor Data}
To thoroughly investigate how much of the excess returns of the anomaly factors are being driven by the aggregate volatility risk factor return generated by the Fama and French factors needs to be ruled out. Data from the Kenneth R. French data library is the source used for Fama and Fench factor data. The Fama and French three-factor (FF3) data set contain the excess return on the market factor $(Rm - Rf)$, the size factor (SMB) and the value factor (HML). The Fama and French five-factor (FF5) data set contain the same factors as the FF3 dataset with the addition of the profitability factor (RMW) and the investment factor (CMA). The momentum factor data, used in the six-factor model, is sourced from WRDS.

\subsubsection{Volatility Data}
The measure of aggregate volatility used for this research is the Chicago Board Options Exchange (CBOE) volatility index (VXO). The VXO is a measure of implied volatility calculated using the 30-day S\&P 100 index (OEX). The VXO index was chosen over the newer VIX index, which is calculated using the S\&P 500 index, because the VXO has a more extended data range and the is highly correlated with the VIX. Price data for the VXO was taken from WRDS.

\subsubsection{Economic Uncertainty Data}
The index for economic policy uncertainty (EPU) was developed by\cite{baker2016measuring} and is constructed from three components. The first component is an analysis of the economic sentiment of ten large US-based newspapers. The second component looks at the number of federal tax code provisions set to expire in the next ten years, based on reports from the Congressional Budget Office. The final component looks at the level of disagreement between professional economic forecasters, utilising data from the Federal Reserve Bank of Philadelphia's Survey of Professional Forecasters.

\subsubsection{Market Liquidity Data}
As a proxy for market liquidity\cite{pastor2003liquidity}'s aggregated, liquidity index is used. 


\section{Portfolio Performance: Unconditional Pattern}
\subsection{Excess Return }
The highest mean excess returns in the long arm are generated by the Momentum (0.0133) anomaly, and the lowest mean excess returns in the long arm are produced by the Accruals (0.0098) and O-Score (0.0097) anomalies. In the short arm of the portfolios, the highest mean excess returns are created by the Gross Profitability (0.0080) anomaly, and the lowest is produced by the Momentum (0.0029) anomaly. Finally looking at the long minus short portfolios the greatest mean excess returns is generated by the Momentum (0.0103) anomaly with the lowest mean returns being generated by the O-Score (0.0021) anomaly.

Stock Issuance (0.0412) has the smallest standard deviation in the long arm of the portfolios, while the Momentum (0.0615) and Accrual (0.0578) anomalies have the largest standard deviations. Looking at the short arm of the portfolio, the Net Operating Assets (0.0530) anomaly has the smallest standard deviation, and Failure Probability (0.0871) has the largest standard deviations. Examining the standard deviations of the long minus short portfolios Stock Issuance (0.0291), Net Operating Assets (0.0293) and Investment to Assets (0.0295) all have the smallest standard deviations. At the other end of the scale, the Momentum (0.0692) and Failure Probability (0.0680) anomalies have the largest standard deviations.

Looking at the correlations between each of the anomaly portfolios, there is very little correlation in the returns to each of the portfolios. Composite Equity Issue and Net Stock issue are the most correlated anomalies (0.75) with the Failure Probability and O-Score anomalies having the second-highest correlation (0.7). Both of these high correlations make sense as what the underlying anomalies are measuring is related. The least correlated anomalies are Assets Growth and Momentum (0).

\import{tables/}{uncon-stats}

\subsection{Risk-Adjusted Returns}
\subsubsection{Capital Asset Pricing Model}
The alphas, of all eleven anomalies, are positive in the Capital Asset Pricing Model regression. Looking at the t-statistics in table \ref{tab:Table 2}, nine of the alphas are significant at the 5\% level and can reject the null hypothesis $(\alpha=0)$. The nine significant alphas are Asset Growth (0.0064), Composite Equity Issuance (0.0064), Failure Probability (0.0123), Gross Profitability (0.0052), Investment to Assets (0.0039), Momentum (0.0121), Net Operating Assets (0.0067), Return on Assets (0.0087) and Stock Issuance (0.0067).

Looking at the t-statistics of the market factor the null hypothesis $(\beta_{mkt}= 0)$ can be rejected at the 5\% level of significance for Asset Growth (-2.8917), Composite Equity Issue (-5.4860), Failure Probability (-6.3183), Gross Profitability (-3.7007), Investment to Assets (-2.9314), O-Score (-3.7489), Return on Assets (-4.8412) and Stock Issuance (-5.3211). 

\subsubsection{Fama and French Three-Factor Model}
With the Fama and French three-factor model as with the CAPM, the alphas of all eleven anomalies are positive. However, when looking at the t-statics, Asset Growth (0.0027) and Investment to Assets (1.9801) are no longer significant at the 5\% level meaning that the null hypothesis $(\alpha=0)$ can no longer be rejected. Looking at the t-statistic for O-Score (3.1554) it is now significant at the 5\% level, and the null hypothesis that the alpha equals zero can be rejected. 

Examining the t-statistics of the Fama and French Three Factors all of the same anomalies can reject the null hypothesis for the market factor $(\beta_{mkt}= 0)$ at the 5\% level of significance with the addition of Momentum (-3.0342). 

Looking at the t-statistics for the small minus big factor the null hypothesis of $(\beta_{smb}= 0)$ can be rejected for Accruals (2.6101), Composite Equity Issuance (-5.6413), Failure Probability (-2.5094), Gross Profitability (-2.5718), O-Score (-12.7384), Return on Assets (-8.2358) and Stock Issuance (-4.8079) at the 5\% level of significance. 

For the high minus low factor the null hypothesis $(\beta_{hml}= 0)$ can be rejected, at the 5\% level of significance, for Asset Growth (5.6809), Composite Equity Issuance (3.3905), Gross Profitability (-6.0304), Investment to Assets (2.8756), Momentum (-2.2889), Net Operating Assets (2.6413) and O-Score (-6.0907).

\subsubsection{Fama and French Five-Factor Model}
The alpha generated by the Asset Growth anomaly, in the Fama and French five-factor model, is negative (-0.0003), but it is not significant at the 5\% level (-0.2609). Looking at the alphas of the other ten anomalies, they are all positive, and seven of the ten positive alphas are significant at the 5\% level. The only change from the Fama and French three-factor model is that the t-statistic of the alpha for the Composite Equity Issuance (1.7298) is no longer significant at the 5\% level, meaning that the null hypothesis that it equals zero can not be rejected.

Examining the test statistics of the Fama and French five factors the null hypothesis for the market factor $(\beta_{mkt}= 0)$ is no longer rejected, at the 5\% level of significance) for Asset Growth (-0.0002), Investment to Assets (-0.0004) and Momentum (-1.9813). 

Looking to the size factor the null hypothesis $(\beta_{smb}= 0)$ is no longer rejected for Failure Probability (-1.0408) and Gross Profitability (1.3734) at the 5\% level of significance.

Moving to the high minus low factor the null hypothesis $(\beta_{hml}= 0)$ is no longer rejected, at the 5\% level of significance, for Accruals (0.5756) and Composite Equity Issuance (0.0008). However, the null hypothesis $(\beta_{hml}= 0)$ is no longer rejected for Momentum (-3.6758) and Net Operating Assets (3.2529) at the 5\% level of significance.

Exploring the t-statistics for the robust minus weak factor the null hypothesis of $(\beta_{rmw}= 0)$ can be rejected, at the 5\% level of significance, for Accruals (-3.4581), Composite Equity Issuance (5.5378), Failure Probability (2.7921), Gross Profitability (8.0593), Investment to Assets (-4.1968), Net Operating Assets (-2.5938), O-Score (5.3981), Return on Assets (9.6893) and Stock Issuance (6.4531).

Studying the conservative minus aggressive factor the null hypothesis $(\beta_{cma}= 0)$ can be rejected, at the 5\% level of significance, for Accruals (2.8569), Asset Growth (9.3610), Composite Equity Issuance (5.6771), Failure Probability (2.0627), Gross Profitability (4.1830), Investment to Assets (7.0179), O-Score (-2.8154) and Stock Issuance (6.4510).

\subsubsection{Fama and French Six-Factor Model}
Only six of the eleven alphas are significant at the 5\% level of significance in the Fama and French six-factor model. The Asset Growth alpha is still negative and insignificant, the same as the five-factor model. Looking at the t-statistics, compared to the five-factor model Failure Probability (1.8605) and Gross Profitability (1.7810) are no longer significant however the Accrual anomaly (2.0647) is significant at the 5\% level.

Examining the t-statistics of the Fama and French six factors, there is no change in the significance of the market factor for any of the anomalies between the Fama and French five-factor and six-factor models. Looking at the size factor, its significance for each of the anomalies is the same as the five-factor model except for Failure Probability (-2.0320) which is now significant at the 5\% level of significance. Exploring the t-statistics for the value factor, the significance of the betas for each of the anomalies is the same as the five-factor model, except for Momentum (-0.9102) witch is no longer significant at the 5\% level of significance. Moving to the t-statistics of the robust minus weak factor nine betas are significant, the same nine in the five-factor model. Examining the conservative minus aggressive anomaly, except for the Failure Probability (0.0051) anomaly, which is not significant at the 5\% level, the same betas are significant as in the five-factor model.

Finally looking at the t-statistics for the up minus down factor the null hypothesis $(\beta_{umd}= 0)$ can be rejected, at the 5\% level of significance, for Failure Probability (6.4892), Gross Profitability (3.4074), Momentum (24.9333), Net Operating Assets (2.3091) and Return on Assets (6.6726).

That the up minus down factor almost entirely explains the returns for the Momentum anomaly is due to them measuring the same return aspect, though there are minor differences in the construction of the anomaly and the factor.

\subsubsection{Fama and French Six-Factor Model + Volatility Factor}
With the addition of the volatility factor to the Fama and French six-factor model, none of the alphas generated by the anomaly portfolios is significant, at the 5\% level, meaning that the null hypothesis that the alpha equals zero cannot be rejected for any of the portfolios. 

Examining the t-statistics for each of the factors that carry over from the six-factor model, the significance of each beta is the same except for the high minus low factor for Return on Assets (-1.5824) which is no longer significant at the 5\% level. 

Looking at the volatility factor, the t-statistics for Momentum (2.4256), Net Operating Assets (2.4026), O-Score (2.1152) and Return on Assets (2.0674) are significant at the 5\% level of significance meaning that the null hypothesis $(\beta_{vxo}= 0)$ can be rejected, and their volatility betas are significantly different from zero

\import{tables/}{unconditional}

\section{Portfolio Performance: Conditional Pattern}
After looking at the performance of the anomaly portfolios with an unconditional pattern the analysis was run again using three conditional patterns. The three conditions were built on volatility, economic policy uncertainty and market liquidity. For each of these conditions, the top and bottom 30\% values were calculated and used to filter the anomaly portfolio datasets for years that followed a year where volatility, economic policy uncertainty or market liquidity was in the top or bottom 30\%.

\FloatBarrier
\subsection{Volatilty}
\subsubsection{Excess Returns}
Looking at the excess returns in periods following high volatility, O-Score generates the highest excess returns in the long arm (0.0139) and the short arm (0.0081) of the portfolios. The lowest mean excess returns in the long arm are generated by the Accrual (0.0101) and the Stock Issuance (0.0102) anomalies. Accruals also generate the third-highest excess returns on the short arm following high volatility (0.0072). Looking at the short arm of the portfolios, the lowest mean excess returns are generated by the Net Operating Assets portfolio (0.0025). Moving to the long minus short portfolios Net Operating Assets (0.0092) produces the highest mean excess returns in periods following high volatility. The lowest mean excess returns in the long minus short portfolio are generated by Accruals (0.0029).

Following periods of low volatility, Net Operating Assets generates the highest excess returns in the long arm (0.0101). The worst excess returns in the long arm in periods following low volatility are generated by the Gross Profitability anomaly (0.0056), but Gross Profitability also generates the highest excess returns in the short arm (0.0095).  Momentum generates the worst excess returns in the short arm (0.0025) but has the second-highest mean excess returns in the long arm (0.0095). Looking at the long minus short portfolio in periods following low volatility, the highest mean returns are generated by Momentum (0.0069) and Net Operating Assets (0.0061). The lowest mean excess returns in the long minus short portfolio are generated by Gross Profitability (-0.0038), and negative mean returns are also generated by the O-Score anomaly (-0.0010).

\import{tables/}{vol-stats}

\subsubsection{Risk-Adjusted Return}
\paragraph{Capital Asset Pricing Model}
In months after periods of high volatility, all elven of the alphas are positive and, at the 5\% level of significance, the alphas of Composite Equity Issuance (0.0096), Failure Probability (0.0147), Gross Profitability (0.0108), Net Operating Assets (0.0089), Return on Assets (0.0111) and Stock Issuance (0.0082) are significant. In contrast to this in periods following low volatility the alpha generated by Gross Profitability is negative, though this is not significant at the 5\% level and therefore we cannot reject the null hypothesis that the Net Operating Assets alpha equals zero. The only significant alpha, in periods following low volatility, is Net Operating Assets (0.0062) based on its t-statistic (2.5181) the null hypothesis can be rejected and the alpha for Net Operating Assets is statistically significant form zero. 

Looking at the market factor betas, in months successive to periods of high volatility, seven of the betas are significant at the 5\% level, based on their t-statistics. The anomalies with significant t-statistics are Asset Growth (-2.1172), Composite Equity Issuance (-4.0682), Failure Probability (-7.1187), Gross Profitability (-4.7411), Momentum (-2.4004), Return on Assets (-5.6442) and Stock Issuance (-4.7524), meaning that their market betas are significantly different from zero. Moving to months following periods of low volatility, only five of the anomalies are significant at the 5\% level. In contrast to the periods after high volatility, in the periods after low volatility, the market betas for Asset Growth, Gross Profitability, Momentum and Stock Issuance are not significant at the 5\% level. The t-statistics for Accrual (-2.6450), Composite Equity Issuance (-2.8224), Failure Probability (-3.4798), O-Score (-2.9301) and Return on Assets (-2.8970) are all statistically significant and therefore the null hypothesis $(\beta_{mkt}= 0)$ can be rejected.

Compared to the unconditional CAPM, the alphas for Asset Growth, Investment to Assets and Momentum are not statistically significant in either the high or low volatility conditional models. The market betas for Accrual and Momentum are not statistically significant in the unconditional model, but the Accrual beta (-0.0024) in the low volatility model and the Momentum beta (0.0093) in the high volatility model are statistically significant, at the 5\% level. 

\paragraph{Fama and French Three-Factor Model}
The alpha generated by the anomaly portfolios in periods following either high or low volatility is positive for all of the portfolios. Focusing on the alpha produced following periods of high volatility, eight of the eleven alphas are statistically significant at the 5\% level. The statistically significant high volatility alphas are Asset Growth (0.0067), Composite Equity Issuance (0.0099), Failure Probability (0.0150), Gross Profitability (0.0109), Net Operating Assets (0.0088), O-Score (0.0078), Return on Assets (0.0121) and Stock Issuance (0.0086). The alpha for Net Operating Assets is no longer statistically significant in the low volatility model. Now the only statistically significant low volatility alpha is Failure Probability (0.0077). 

Looking at the anomaly portfolios in high and low volatility models the Accrual anomaly has only a significant size beta (-0.0026) in periods of high volatility and only a significant value beta in the low volatility model. Asset Growth and Composite Equity Issuance load significantly on all three factors in the high volatility model but only significantly load on the value and size factors respectively in the periods following low volatility. Failure Probability only has a significant market beta and Return on Assets has significant market and size betas in periods after high volatility, but both anomalies have a significant market, size and value beta in periods of low volatility. Gross Profitability has a significant market and value beta in both conditional volatility models. Investment to Assets has no significant factors in either the high or the low volatility models. Momentum only has a significant market beta in the high volatility model and nothing in the low volatility model. On the other hand, Net Operating Assets has no significant betas in the model of periods following high volatility but does have significant betas for size and value in the low volatility model. O-Score has a significant size, and value factor and Stock Issuance has a significant size and market factor in the high volatility model. Both O-Score and Stock Issuance have significant size betas in the low volatility model.

Comparing the alphas in the conditional model to the unconditional model, Momentum is not statistically significant in either the high or low volatility models, but it is significant in the unconditional model. The alpha for Asset Growth is not statistically significant in the unconditional model, but it is in the high volatility model.

\paragraph{Fama and French Five-Factor Model}
All of the alphas generated by the anomaly portfolios in periods following high volatility are positive though only Composite Equity Issuance (0.0059), Net Operating Assets (0.0103) and O-Score are statistically significant at the 5\% level. Looking at the alphas produced in months following low volatility they are all positive except for the O-Score portfolio (-0.0016) however looking at the corresponding t-statistic (-0.7067) it is not significant at the 5\% level. The only portfolio with a significant alpha in the low volatility model is Investment to Assets (0.0051).

Comparing the loading of the anomaly portfolios onto the five Fama and French factors in the high and low volatility models, in the high volatility model Accruals loads onto the size, profitability and investment factors and Momentum loads onto the market and value factors but neither of them has any significant betas in the low volatility model. Asset Growth has a significant size and investment beta in the high volatility model but only has a significant investment beta in the low volatility model. In the high volatility model, Composite Equity Issuance has a significant market, profitability and investment beta and O-Score has significant betas for all of the factors. In the low volatility model, both Composite Equity Issuance and O-Score have significant size, profitability and investment betas. Failure Probability has a significant market beta, Stock Issuance has a significant investment beta, and both have significant value and profitability betas in the model of periods following high volatility. In the low volatility model, both Failure Probability and Stock Issuance have significant size and profitability betas. Gross Profitability has a significant market, value, profitability and investment beta in both the model of periods following high volatility and the model of periods following low volatility. In the high volatility model Investment to Assets has only a significant investment beta but in the low volatility model Investment to Assets has significant loading on both the size and investment betas. Net Operating Assets has no significant betas in the high volatility model though in the low volatility model, it does significantly load onto the size, value and profitability factors. In the model of the months after high volatility, Return on Assets has a significant market, size, value, and profitability beta but in the low volatility model, only the size and profitability betas are significant. 

In comparison with the unconditional FF5 model, the alpha generated by Failure Probability, Gross Profitability, Momentum, Return on Assets and Stock Issuance is not statistically significant in either the high or low volatility models. In the unconditional FF5 model neither Composite Equity Issuance or Investment to Assets have a statistically significant alpha, but in the high and low volatility models respectively Composite Equity Issuance and Investment to Assets generate statistically significant alpha. 

\paragraph{Fama and French Six-Factor Model}
The alphas of the anomaly portfolio in the high volatility model are all positive with Composite Equity Issuance (0.0059), Momentum (0.0063), Net Operating Assets (0.0103), O-Score (0.0074) and Return on Assets (0.0060) also being statistically significant at the 5\% level meaning that these alphas are significantly different from zero. Moving to the low volatility model, the alphas generated by nine of the portfolios are positive. In addition to the low volatility alpha for the O-Score portfolio (-0.0008) being negative, Momentum (-0.0015) and Return on Assets (-0.0011) also generate a negative alpha in periods following low volatility. Although the estimates of the low volatility alphas are negative, they are not significant and therefore the null hypothesis, that they equal zero, cannot be rejected. The only statistically significant alpha for the low volatility portfolio is Stock Issuance (0.0039).

In the high volatility six-factor model, the Accrual anomaly has a significant size, profitability and investment beta but in the low volatility model, none on the betas are significant for the Anomaly portfolio. Asset Growth has a significant size and investment beta in the model of periods following high volatility and only a significant investment beta in the low volatility model. Composite Equity Issuance and Stock Issuance both have significant profitability, and investment betas in the high volatility model and both have significant size, profitability and momentum betas in the low volatility model; though Composite Equity Issuance also has a significant investment beta. Failure Probability has significant loading on the market, profitability and momentum factors in the model of months following high volatility and significant loading on the market, size, profitability and momentum factors in the model of low volatility. Gross Profitability has significant market, value, profitability and investment betas in both the models of high and low volatility. Also, Gross Profitability has a significant momentum beta in the model of periods following high volatility. In the model of months after a period of high volatility Investment to Assets only has a significant investment beta, but in the low volatility model, the size, profitability and momentum betas are also significant. For both the model of periods following high volatility and the model of periods following low volatility, Momentum only has a significant momentum beta. Net Operating Assets has significant loading on both the value and momentum factors in the high volatility model and the model of periods following low volatility, the value factor has significant loading, along with the additions of size and profitability, but the momentum factor does not. In the model of periods following high volatility, O-Score has significant size and profitability betas, and in the model of periods after low volatility, it also has significant investment and momentum betas, in addition to the significant size and profitability betas. Return on Assets has significant loading on the market, size, value, profitability and momentum factors in the model of periods following high volatility but for the model of periods following low volatility, there is only significant loading on profitability and momentum.

Looking back to the unconditional FF6 model, the alpha for the Accrual is significant in the unconditional model, but it is not significant in either the high or low volatility conditional models. The alpha for Composite Equity Issuance is not significant in the unconditional model, but it is in the high volatility conditional model. 

\paragraph{Fama and French Six-Factor Model + Volatility Factor}
There are two negative alphas in the high volatility model and seven in the low volatility model. The only statistically significant alpha in periods following high volatility is Stock Issuance (0.0230) meaning that for all the other alphas in the high volatility model the null hypothesis, that they are equal to zero, cannot be rejected. Likewise, the only statistically significant alpha for months after periods of low volatility is O-Score (-0.0244), meaning again for the rest of the alphas in the low volatility model the null hypothesis cannot be rejected.

Examining the t-statistics for the volatility factor the only statistically significant one in the high volatility model is Stock Issuance (-2.7411), and the only significant t-statistic in the low volatility model is O-Score (2.6981), for each of these the null hypothesis, that the volatility beta equals zero can be rejected. These significant volatility betas align with the two anomaly portfolios that have statistically significant alphas.

Comparing the models looking at periods following high and low volatility, the Accruals anomaly has a significant size, profitability and investment beta in the model of periods following high volatility and the Accrual portfolio has no significant betas in the model of periods following low volatility. Asset Growth has significant loading on both the size and investment factors in the high volatility model but only has a significant load on the investment factor in the low volatility model. Composite Equity Issuance has a significant market, profitability and investment beta in the model of periods following high volatility and in the model of the periods following low volatility Composite Equity Issuance has a significant loading on the size, profitability, investment and momentum factors. In the high volatility model, Failure Probability has significant loading on the market, profitability and momentum factors whereas in the low volatility model Failure Probability has significant loading on the size, profitability and momentum factors. Gross Profitability has, in the high volatility model, significant market, value, profitability, investment and momentum betas but in low volatility model, Gross Profitability only has a significant market, value, profitability and investment beta. In the model of periods following high volatility, Investment to Assets only has significant loading on the investment factor although in the model of periods following low volatility Investment to Assets has significant loading on the size, profitability, investment and momentum factors. Momentum only has a significant momentum beta in both the model of periods following high volatility and the model of periods following low volatility. Net Operating Assets has, in the model of periods following high volatility, significant value and momentum betas. In the model of periods following low volatility, Net Operating Assets has a significant size, value and profitability beta. O-Score has significant loading on size and profitability in the model of periods following high volatility. For the model of periods following low volatility, O-Score has significant loading on the size, profitability, investment and volatility factors. Return on Assets has, in the model of months after high volatility, significant market, size, profitability and momentum betas but in the low volatility model Return on Assets only has significant profitability and momentum betas. Finally, Stock Issuance has significant loading on the market, value, profitability, investment and volatility factors in the high volatility model though in the model of periods after low volatility Stock Issuance only has significant loading on the size, profitability and momentum factors.

Examining the differences in the conditional and unconditional models, the unconditional model does not have any significant alphas, whereas Stock Issuance and O-Score have significant alphas in the high and low volatility conditional models, respectively. Looking at the volatility factor in the unconditional model, it is significant for the Momentum, Net Operating Assets and Return on Assets portfolios where it is not in either of the conditional models. The volatility beta for Stock Issuance is not significant in the unconditional model, whereas it is in the high volatility conditional model. O-Score has a significant volatility beta in both the unconditional and high volatility conditional model.

\import{tables/}{vol-models}

\FloatBarrier
\subsection{Economic Policy Uncertainty}
\subsubsection{Excess Returns}
In periods following high economic policy uncertainty Asset Growth (0.0179) and Investment to Assets (0.0178) generate the highest mean excess returns in the long arm. The lowest mean excess returns in the long arm are generated by Failure Probability (0.0126). Looking at the short arm, in periods following high economic policy uncertainty, O-Score (0.0155) and Accruals (0.0153) generate the highest mean excess return. Stock Issuance creates the lowest excess returns in the short arm (0.0100). Moving to the long minus short portfolios, following periods of high economic policy uncertainty, the greatest mean excess returns are produced by Investment to Assets (0.0052), and the worst mean excess returns are generated by O-Score (-0.0013).

Examining periods following low economic policy uncertainty Composite Equity Issuance (0.107), Failure Probability (0.0106), Gross Profitability (0.0106) and Return on Assets (0.0104) generate the highest mean excess returns in the long arm. Moving to the lowest mean excess returns in the long arm, these are generated by Accruals (0.0043) and Investment to Assets (0.0051). In periods following low economic policy uncertainty, the highest mean excess returns, in the short arm, are generated by Gross Profitability (0.0071). The lowest mean excess returns are generated by Failure probability (-0.0031). Momentum (-0.0020), Return on Assets (-0.0011) and O-Score (-0.0002) all also produce negative mean excess returns in the short arm following low economic policy uncertainty. Finally looking at the long minus short portfolios, the highest mean excess returns are generated by Failure Probability (0.0137) with the lowest mean excess returns being produced by Investment to assets (0.0000).

\import{tables/}{epu-stats}

\subsubsection{Risk-Adjusted Return}
\paragraph{Capital Asset Pricing Model}
Looking at the alpha generated by the anomalies, in periods following both high and low economic policy uncertainty, it is positive for all of the portfolios. In times after high economic policy uncertainty, the alpha generated by the Composite Equity Issuance (0.0077), Failure Probability (0.0133), Gross Profitability (0.0089), Investment to Assets (0.0067), Return on Assets (0.0078) and Stock Issuance (0.0088) anomalies is statistically significant at the 5\% level of significance. As a result of the statistical significance, the null hypothesis that the alphas are equal to zero can be rejected. Looking at periods following low economic uncertainty, the alpha produced by the Composite Equity Issuance (0.0078), Failure Probability (0.0155), Net Operating Assets (0.0097), O-Score (0.0096), Return on Assets (0.0126) and Stock Issuance (0.0070) anomalies is significantly different from zero, at the 5\% level of significance. 

Examining the factor loadings in the two economic policy uncertainty CAPMs, the Accrual and Investment to Assets portfolios do not significantly load on the market factor in either the high or low uncertainty models. On the other hand, the Composite Equity Issuance, Failure Probability and Stock Issuance portfolios have significant market betas in both the high and low economic policy uncertainty models. The Gross Profitability, Momentum, Net Operating Assets, O-Score and Return on Assets anomalies only have significant betas in the high uncertainty model while Asset Growth only has a significant market beta in the low uncertainty model.

In comparison to the unconditional CAPM, the alpha generated by the anomalies Asset Growth and Momentum is not significant in either of the conditional economic uncertainty models. The alpha generated by the O-Score anomaly is not significant in the unconditional model, but it is significant in the low economic policy uncertainty model

\paragraph{Fama and French Three-Factor Model}
The alphas generated by all the portfolios in the high economic uncertainty Fama and French three-factor model are positive, and the alpha generated by the Composite Equity Issuance (0.0084), Failure Probability (0.0134), Gross Profitability (0.0079), Investment to Assets (0.0070), Return on Assets (0.0077) and Stock Issuance anomalies is also significant at the 5\% level. The statistical significance of these alphas means that the null hypothesis that the alpha is equal to zero can be rejected. Analysing the alphas generated by the low economic uncertainty model they are all positive except for Investment to Assets (-0.0001) but its t-statistic (-0.0267) is too small to reject the null hypothesis which is that it is equal to zero. Looking at the rest of the alphas produced in the low economic policy uncertainty model those generated by the Failure Probability (0.0154), Gross Profitability (0.0066), Momentum (0.0162), Net Operating Assets (0.0086), O-Score (0.0078) and Stock Issuance (0.0051) anomalies are significantly different from zero. 

Comparing the high and low economic policy uncertainty three-factor models, the Accrual anomaly has no significant betas in the high uncertainty model and significant market and size betas in the low uncertainty model. In the model of periods following high economic policy uncertainty, Asset Growth only significantly loads on the value factor whereas in the model of periods following low economic uncertainty there is significant loading on the market factor as well as the value factor. The Composite Equity Issuance portfolio has significant market and size factor loadings in the model of high economic uncertainty and has significant loading on all three factors in the model of periods following low economic uncertainty. On the other hand, the Failure Probability and Gross Profitability portfolios load significantly on all three factors in the model of periods after high economic uncertainty, but in the low uncertainty model, there is only significant loading on the market factor. Investment to Assets has only a significant value beta in the high uncertainty model and no significant betas in the low economic policy uncertainty model. In the high economic policy uncertainty model, Momentum has a significant market beta and a significant value beta in the low economic policy uncertainty model. The Net Operating Assets portfolio loads significantly on the value factor in the high uncertainty model and on the size factor in the low uncertainty model. In both the economic policy uncertainty models, the O-Score portfolio loads significantly on the size and value factors. In the high economic policy uncertainty model, the Return on Assets portfolio loads significantly on all three factors and only on the size factor in the low economic policy uncertainty model. Stock Issuance has significant market and size betas in the high economic policy uncertainty model, whereas in the low economic policy uncertainty model, only the market beta is significant. 

Looking back to the unconditional three-factor model, the Investment to Assets alpha is not significant, but it is in the high economic policy uncertainty conditional model. All of the other alphas that are significant in the unconditional model are significant in at least one of the conditional models.

\paragraph{Fama and French Five-Factor Model}
Examining the alphas generated by the anomaly portfolios in the high economic policy uncertainty model one of them, Momentum (-0.0002), is negative. However, looking at the t-statistics, none of the alphas are significant, and therefore none can reject the null hypothesis that the alpha is equal to zero. Moving to look at the alpha produced by the low economic policy uncertainty model both the Asset Growth (-0.0006) and the Investment to Assets (-0.0000) anomalies are negative, but neither is statistically significant at the 5\% level meaning that the null hypothesis that they are zero cannot be rejected. Analysing the rest of the low economic policy uncertainty alphas those generated by the Momentum (0.0154), Net Operating Assets (0.0089) and O-Score (0.00063) anomalies are significantly different from zero. 

Looking at the factor loadings on the two economic policy uncertainty models, in the high uncertainty model the Accruals portfolio loads significantly on the profitability factor and the market and size factors in the low economic policy uncertainty model. Asset Growth has a significant size and investment beta in the high uncertainty model but just a significant investment beta in the low economic policy uncertainty model. In the model of periods following high economic policy uncertainty Composite Equity Issuance loads significantly on the market, size, profitability and investment factors and the size, profitability and investment factors in periods following low economic policy uncertainty. Failure Probability has, in the high uncertainty model, significant market, value, profitability and investment betas and, in the low uncertainty model, significant market, value and profitability betas. For the high economic policy uncertainty model the Gross Profitability anomaly loads significantly on all five of the factors except for size and for the model of periods following low economic policy uncertainty Gross Profitability loads significantly on all of the factors. Investment to Assets only has a significant investment beta in the model of periods following high economic policy uncertainty whereas in the model of periods following low economic policy uncertainty the size, profitability and investment betas are significant. In the high and low economic policy uncertainty models Momentum loads significantly on the profitability and investment factors and also on the value factor in the low uncertainty model. Net Operating Assets has a significant value beta in the high uncertainty model and a significant size beta in the low uncertainty model. The O-Score portfolio has significant loading on the size and investment factors in the high economic policy uncertainty model whereas in the low uncertainty model there is significant loading on all of the factors except for the market factor. In the model of periods following high economic policy uncertainty, Return on Assets has significant market, size and profitability betas but only significant size and profitability betas in the low economic policy uncertainty model. The Stock Issuance portfolio has significant loading on all of the factors except for value in the high economic policy uncertainty model whereas in the low economic uncertainty model there is only significant loading on the profitability and investment factors.

Comparing the alphas generated by the Failure Probability, Gross Profitability, Stock Issuance anomalies they are significant in the unconditional five-factor model but are not significant in either the high or the low economic policy uncertainty conditional models. The alphas generated by Momentum, Net Operating Assets and O-Score are significant in both the unconditional model and the low economic policy uncertainty conditional model. Asset Growth has significant size and investment betas in the high uncertainty model and a significant investment beta in the low 

\paragraph{Fama and French Six-Factor Model}
Studying the alphas generated by the high economic policy uncertainty six-factor model they are all positive with the alphas produced by Failure Probability (0.0101) and Momentum (0.0079) also being statistically significant. The statistical significance of the alpha from Failure Probability and Momentum allows for the rejection of the null hypothesis that they equal zero. Moving over to the alphas in the low economic policy uncertainty model the alphas from Asset Growth (-0.0005), Investment to Assets (-0.0009) and Return on Assets (-0.0015) are all negative but none are statistically significant. The only alpha in the low economic policy uncertainty model that is significant is generated by the O-Score anomaly (0.0062). 

Comparing the factor loadings between the two economic policy uncertainty models, the Accrual portfolio has significant loading on the profitability and investment factors in the high uncertainty model and on the market and size factors in the low uncertainty model. In the model of periods after high economic policy uncertainty, Asset Growth has a significant size and investment beta but only a significant investment beta in the low uncertainty model. Composite Equity Issuance has, in the model of periods following high economic uncertainty, significant loading on the market, size, profitability and investment factors and significant loading on size, profitability and investment in the low economic policy uncertainty model. The Failure Probability anomaly loads significantly on the market size, profitability and investment factors in the high uncertainty model and on the market, profitability and momentum factors in the low uncertainty model. Gross Profitability has significant loading on all of the factors except size in the model of high economic policy uncertainty and on all of the factors except momentum in the model of low economic policy uncertainty. In the model of periods following high economic policy uncertainty, Investment to Assets has only a significant investment beta but in the model of periods following low economic policy uncertainty the size and profitability betas are also significant. In both of the models of economic policy uncertainty, the Momentum portfolio has a significant size and momentum beta. Net Operating Assets only significantly loads on the value factor in the model of periods after high economic policy uncertainty, but in the model of low economic policy uncertainty, there is significant loading on both the size and momentum factors. The O-Score portfolio has a significant size and investment beta in both the high and low models of economic policy uncertainty, however, the profitability beta is also significant in the low economic policy uncertainty model. In both the high and low models of economic policy uncertainty, Return on Assets has significant loading on the size, profitability and momentum factors and there is also significant loading on the investment factor in the model of periods following high economic policy uncertainty. The Stock Issuance portfolio has a significant market, size, profitability and investment beta in the model of periods following high economic policy uncertainty but only significant profitability and investment betas in the model of periods following low economic policy uncertainty.

Looking back to the unconditional six-factor model, the Accrual, Net Operating Assets, Return on Assets and Stock Issuance anomalies alphas are significant, but they are not significant in either of the conditional economic policy uncertainty models. The alphas generated by Momentum and O-Score are significant in both the unconditional model and, respectively, in the high and low economic policy uncertainty conditional models. Failure Probability only generates significant alpha in the high economic policy uncertainty conditional model.

\paragraph{Fama and French Six-Factor Model + Volatility Factor}
Gross Profitability (-0.0049), O-Score (-0.0071) and Return on Assets (-0.0032) all report negative alphas in the high economic policy uncertainty model, but none of these alphas is statistically significant. Therefore the null hypothesis that the alphas, for Gross Profitability, O-Score and Return on Assets, equal zero cannot be rejected. The only statistically significant alpha in the high economic policy uncertainty model is generated by Stock Issuance (0.0144). Examining the alphas generated by the low economic policy uncertainty model those generated by the Composite Equity Issuance  (-0.0056), Momentum (-0096), Net Operating Assets (-0.0079) and O-Score (-0.0070) are negative but not significantly different from zero. However, the alpha generated by the Accrual anomaly (-0.0160) is both negative and significant at the 5\% level, and as a result, the null hypothesis that it is equal to zero can be rejected. The only other statistically significant alpha, in the low economic policy uncertainty model, is generated by Gross Profitability (0.0107).

None of the volatility betas in the high economic uncertainty model is statistically significant at the 5\% level, and therefore none reject the null hypothesis that they are equal to zero. Looking at the low economic uncertainty model, the only anomaly with a statistically significant volatility beta is the Accrual anomaly (0.0009), which also has a significant alpha.

Examining the loadings on the factors between the two models of economic policy uncertainty, the Accrual portfolio has significant loading on the profitability and investment factors in the model of periods following high economic policy uncertainty and significant loading on the size and volatility factors in the model of periods following low economic policy uncertainty. Asset Growth has a significant size and investment beta in the high uncertainty model but only a significant investment beta in the low uncertainty model. In the high uncertainty model, Composite Equity Issuance has significant loading on the market, size, profitability and investment factors but in the low uncertainty model, there is just significant loading on the profitability and investment factors. For both models of economic policy uncertainty, the Failure Probability portfolio has a significant market, profitability and momentum beta and in the high uncertainty model, the size beta is significant as well. Gross Profitability has significant loading on all of the six Fama and French factors in both models except for size in the high uncertainty model and momentum in the low uncertainty model. In the model of periods following high economic policy uncertainty the Investment to Assets anomaly only has a significant investment beta but in the model of periods following low economic policy uncertainty the size and profitability betas are significant as well. The Momentum portfolio has significant loading on both the size and momentum factors in both the high and low economic policy uncertainty models. Value is the only significant beta for Net Operating Assets in the model of periods following high economic policy uncertainty, but in the model of periods following low economic policy uncertainty, the size and momentum betas are significant. The O-Score anomaly has significant loading on the size and investment factors in both models of economic policy uncertainty, but in the low uncertainty model, there is also significant loading on the profitability factor. In both models of economic policy uncertainty, Return on Assets has a significant size, profitability and momentum beta and the high uncertainty model also has a significant investment beta. The Stock Issuance portfolio has significant loading on the market, size, value, profitability and investment factors in the model of periods following high economic policy uncertainty but in the model of periods following low economic policy uncertainty, there is only significant loading on the profitability and investment factors.

Comparing back to the unconditional six-factor plus volatility model none of the anomaly alphas is statistically significant whereas in the high economic policy uncertainty model the alpha generated by Stock Issuance is and in the low economic policy uncertainty model both Accruals and Gross Profitability have significant alpha. Looking at the volatility betas in the unconditional model it is significant for Momentum, Net Operating Assets, O-Score and Return on Assets none of which have a significant volatility beta in either of the conditional economic policy uncertainty models. On the other hand, the volatility beta for the Accrual anomaly is significant in the low economic policy uncertainty model whereas it is not in the unconditional model.

\import{tables/}{epu-models}

\FloatBarrier
\subsection{Market Liquidity}
\subsubsection{Exess Returns}
Following periods of high market liquidity, the highest mean excess returns in the long arm are generated by Momentum (0.0225). However, Momentum also generates the lowest mean returns in the short arm (0.0098). The worst mean excess returns in the long arm is produced by Asset Growth (0.0130). Looking to the short arm the highest mean excess returns, following periods of high market liquidity, are generated by Accruals (0.0177). Examining the mean excess returns in of the long minus short portfolios in periods following high market liquidity the highest returns are generated by Momentum (0.0126) with the lowest returns being produced by Accruals (-0.0027) and Asset Growth (-0.0013)

In periods following low market liquidity the highest mean excess returns, in the long arm, is produced by Composite Equity Issuance (0.0094) while the lowest mean returns are generated by Momentum (0.0017) and Accruals (0.0021). Examining the short arm returns, in periods following low market liquidity, the highest mean excess returns are delivered by Gross Profitability (0.0016) and Investment to Assets (0.0014). The worst mean excess returns in the short arm are created by Failure Probability (-0.0083). Negative mean excess returns are also delivered by Net Operating Assets (-0.0048), Return on Assets (-0.0044), Accruals (-0.0034), Stock Issuance (-0.0020), O-Score (-0.0018), Momentum (-0.0006) and Composite Equity Issuance (-0.0004) in the short leg. Finally looking at the long minus short portfolios, following periods of low market liquidity, Failure Probability (0.0123) generates the highest mean excess returns while Momentum (0.0023) produces the lowest.

\import{tables/}{liq-stats}

\subsubsection{Risk-Adjusted Return}
\paragraph{Capital Asset Pricing Model}
The alphas generated by the anomaly portfolios, in months following high market liquidity, are all positive except for the Accrual anomaly (-0.0008) but the t-statistic for the Accruals alpha is not large enough to reject the null hypothesis that it is equal to zero. Significant alphas are generated by Composite Equity Issuance (0.0056), Momentum (0.0143), Net Operating Assets (0.0068), Return on Assets (0.0076) and Stock Issuance (0.0064) in periods following high market liquidity. Moving to months after low market liquidity all of the alphas are positive with those generated by the Composite Equity Issuance (0.0103), Failure Probability (0.0139), Net Operating Assets (0.0111), Return on Assets (0.0101) and Stock Issuance (0.0096) anomalies also being significant at the 5\% level. Anomalies with a significant alpha can reject the null hypothesis that the alpha is equal to zero.

The market beta is not significant for the Accruals and Net Operating Assets portfolios in either the high or the low liquidity models. On the other hand, the market beta is significant in both the high and low liquidity models for the Asset Growth, Composite Equity Issuance, Failure Probability and Stock Issuance portfolios. The market beta is significant only in the model of periods following low volatility for the Gross Profitability, Investment to Assets, Momentum, O-Score and Return on Assets portfolios.

Comparing back to the unconditional model, the Asset Growth, Gross Profitability and Investment to Assets portfolios alphas were significant whereas they are not in either of the conditional liquidity models. In the unconditional model, Momentum does not have a significant market beta, but it does in the model of periods following low market liquidity. 

\paragraph{Fama and French Three-Factor Model}
The alpha generated by the Accruals anomaly (-0.0007), in the high liquidity model, is the only alpha in either of the two liquidity three-factor models that is negative. However, looking at the t-statistic (-0.2624) it is not significant at the 5\% level and therefore, the null hypothesis that the Accrual alpha equals zero cannot be rejected. Significant alpha is generated, in the model of periods following high liquidity, by the Composite Equity Issuance (0.0062), Failure Probability (0.0118), Gross Profitability (0.0082), Momentum (0.0157), Net Operating Assets (0.0063), Return on Assets (0.0097) and Stock Issuance (0.0074) anomaly portfolios. On the other hand, in the model of periods following low market liquidity, significant alpha is generated by Composite Equity Issuance (0.0077), Failure Probability (0.0144), Net Operating Assets (0.0104), O-Score (0.0044), Return on Assets (0.0085) and Stock Issuance (0.0083). 

Comparing the high and low liquidity models, the Accrual portfolio does not significantly load on any of the three-factors in the high liquidity model whereas in the low liquidity model the Accrual portfolio loads significantly on the size factor. Asset Growth has a significant market and value beta in the high liquidity model and significant market, size and value betas in the low liquidity model. Composite Equity Issuance has, in the high liquidity model, significant loading on the market and size factors and in the low liquidity model Composite Equity Issuance loads significantly on all three-factors. In the high liquidity model, Failure Probability significantly loads on all three factors, but in the low liquidity model, it only significantly loads on the market factor. Gross Profitability has a significant market and value beta in both the high and low liquidity models. 

Looking back to the unconditional three-factor model significant alpha is generated by the O-Score portfolio but looking at the conditional models O-Score does not generate significant alpha in the high liquidity model though it does in the low liquidity model. Significant alpha is not generated, in the low liquidity model, by the Gross Profitability and Momentum portfolios, but it is in the unconditional model and the high liquidity model.

\paragraph{Fama and French Five-Factor Model}
The only negative alpha generated in either the high or low five-factor liquidity models is in the high liquidity Asset Growth portfolio, though it is not significant at the 5\% level meaning that the null hypothesis that it is equal to zero cannot be rejected. Significant alpha in the high liquidity model is only generated by Net Operating Assets (0.0068) and Stock Issuance (0.0042). In the low liquidity model, significant alpha is generated by Net-Operating Assets (0.0114) and O-Score (0.0057). These significant alpha have large enough t-statistics that the null hypothesis can be rejected and the corresponding alpha is therefore significantly different from zero.

Comparing the factor loadings between the high and low liquidity models the Accrual Portfolio has significant loading on only the size factor in the high liquidity model but in the low liquidity model Accrual loads significantly onto the size, profitability and investment factors. Asset Growth has just a significant investment beta in the model of periods following high market liquidity, but in the model of periods following low market liquidity, it has a significant size beta, as well as a significant investment beta. In months after high liquidity Composite Equity Issuance loads significantly onto the market, size, profitability and investment factors whereas in the model of low liquidity Composite Equity Issuance loads significantly onto the value, profitability and investment factors. Failure Probability has, in the model of high liquidity, significant loading on all five factors, however, in the model of low liquidity, Failure Probability only has significant loading on the market, value and profitability factors. In the model of periods following high liquidity, Gross Profitability has significant value, profitability and investment betas, whereas, in the model of periods following low liquidity, all five of the betas are significant. Investment to assets only has a significant investment beta in the high liquidity model but picks up a significant profitability beta, on top of a significant investment beta, in the low liquidity model. Momentum has significant loading on the value and investment factors in the model of periods following high market liquidity whereas in the model of the periods following low market liquidity Momentum significantly loads onto the market and value factors. In the model of high market liquidity, Net Operating assets has only a significant value beta, but in the model of low market liquidity, the significant value beta is joined by a significant profitability beta. The O-Score portfolio has a significant size and profitability beta in the high liquidity model, whereas, in the low liquidity model, O-Score has a significant market, size, profitability and investment beta. Return on Assets loads significantly on the size, value and profitability betas in the high liquidity model and on the market and profitability factors in the low liquidity model. In the model of periods following high market liquidity Stock Issuance has a significant size, value, profitability and investment beta and in the model of periods following low market liquidity the Stock Issuance portfolio has a significant market, profitability and investment beta. 

Looking back to the unconditional five-factor model, the Failure Probability, Gross Profitability, Momentum, Net Operating Assets, O-Score, Return on Assets and Stock Issuance portfolios generated significant alpha. However, in the high liquidity model, only the alpha generated by Net Operating Assets and Stock Issuance is significant and therefore, different from zero. On the other hand, in the model of periods following low market liquidity, the only significant alphas are generated by the Net Operating Assets and O-Score portfolios. 

\paragraph{Fama and French Six-Factor Model}
Looking at the alphas generated by the anomaly portfolios in both the high and low liquidity six-factor models, the only negative alpha is generated by the Asset Growth portfolio in the high liquidity model (-0.0016). However, the alpha generated in the high liquidity model by the Asset Growth anomaly is not significant at the 5\% level and therefore, the null hypothesis that it equals zero cannot be rejected. In the model of periods following high market liquidity, significant alpha is generated by the Composite Equity Issuance (0.0039), Net Operating Assets (0.0062) and Stock Issuance (0.0041) portfolios. Significant alpha is generated, in the low liquidity model, by the Failure Probability (0.0111), Momentum (0.0079), Net Operating Assets (0.0118), O-Score (0.0056) and Return on Assets (0.0071) anomalies. 

Comparing the factor loadings between the high and low volatility six-factor models, the Accrual portfolio only has significant loading on the size factor in the high volatility model but in the low volatility model Accrual loads significantly on the profitability, investment and momentum factors as well as the size factor. Asset Growth has, in the model of periods following high market liquidity, only a significant investment beta whereas in the model of periods of following low market volatility Asset Growth has a significant investment beta along with the significant size beta. In the model of high liquidity Composite Equity Issuance loads significantly on the market, size, profitability and investment factors and in the model of low liquidity Composite Equity Issuance loads significantly on the value, profitability and investment factors. Failure Probability has a significant market, size, profitability and momentum beta in periods following high liquidity but in the model of low liquidity the size beta is not significant, leaving only the significant market, profitability and momentum betas. In the model of months after periods of high market liquidity, the Gross Profitability portfolio loads significantly on the value, profitability and investment factors whereas in the model of periods following low market liquidity Gross Profitability loads significantly on all six factors. Investment to Assets has significant profitability and investment betas in both the models of periods following high and low market liquidity. Likewise, Momentum loads significantly on only the momentum factor in both models. In the model of periods following high liquidity Net Operating Assets has only ha significant value beta but in the model of periods following low liquidity, it has both a significant value beta and a significant profitability beta. The O-Score portfolio has significant loading on the size, value and profitability factors in the high liquidity model and significant loading on the market, size, profitability and investment factors in the low liquidity model. In the model of periods following high market liquidity, Return on Assets has a significant size, value, profitability and momentum beta whereas in periods following low market liquidity the Return on Assets portfolio has a significant market, size, profitability and momentum beta. Finally, Stock Issuance has, in the high liquidity model, significant loading on the size, value, profitability and investment betas but in the model of periods following low liquidity, the Stock Issuance portfolio only has significant loading on the market, profitability and investment factors. 

In the unconditional six-factor model, the Accrual, Momentum, Net Operating Assets, O-Score, Return on Assets and Stock Issuance portfolios have an alpha that is significantly different from zero. In the model of periods following high market liquidity, the Accrual, Momentum, O-Score and Return on Assets anomalies do not generate significant alpha, as they do in the unconditional model. However, Net Operating Assets and Stock Issuance generate significant alpha, in both the high liquidity and unconditional models, and also Composite Equity Issuance generates significant alpha in the high liquidity model whereas it is not significant in the unconditional model. Looking at the low liquidity model, again the Accrual, Momentum and O-Score portfolios do not generate significant alpha as they do in the unconditional model though unlike the high volatility model Return on Assets does generate significant alpha while Stock Issuance does not. Also, the Failure Probability portfolio generates significant alpha in the low liquidity model but not in either of the unconditional or high liquidity models.

\paragraph{Fama and French Six-Factor Model + Volatility Factor}
Examining the alphas generated in the liquidity models the alpha generated by the Composite Equity Issuance (-0.0046), Gross Profitability (-0.0051), Net Operating Assets (-0.0005), O-Score (-0.0060) and Stock Issuance (-0.0008) anomalies in the high liquidity model and by the Asset Growth (-0.0060), Failure Probability (-0.0015), Gross Profitability (-0.0048), Momentum (-0.0056), Net Operating Assets (-0.0031), O-Score (-0.0007) and Return on Assets (-0.0043) portfolios in the low liquidity model are negative. However, none of the negative alphas is significant, and none of the alphas in either of the six-factor plus volatility liquidity models is significant at the 5\% level. The lack of significance in the alphas means that none of the portfolios can reject the null hypothesis that their alpha is equal to zero. 

None of the volatility betas in the model of periods following high market liquidity is significant at the 5\% level and is therefore not significantly different from zero. Looking at the model of periods following low market liquidity, the only significant volatility beta is generated by the Net Operating Assets portfolio (0.0006). 

Looking at the differences in the factor loadings between the two liquidity models, the Accrual loads exclusively on the size factor in the high liquidity model whereas in the low liquidity model it also loads on the profitability, investment and momentum factors, as well as the size factor. Asset Growth has only a significant investment beta in the high liquidity model, but in the low liquidity model, it has both significant size and investment betas. In the model of periods following high market liquidity, the Composite Equity Issuance portfolio has a significant market, size and profitability factor loading and in the low liquidity model, Composite Equity Issuance has significant value, profitability and investment factor loadings. The Failure Probability portfolio has a market, size, profitability and momentum beta in the high liquidity model which are significant whereas in the low liquidity model Failure Probability does not have a significant size beta leaving it with only significant market, profitability and momentum betas. For the model of periods following high market liquidity, the Gross Profitability anomaly has significant loading on the value, profitability and investment factors and in the model of periods following low market liquidity, Gross Profitability has significant loading on all six of the Fama and French factors but not on the volatility factor. Investment to Assets has, in the model of high liquidity, only a significant investment beta and in the model of low liquidity, the significant investment beta is joined by a significant profitability beta. Again, Momentum loads exclusively on the momentum factor in both the models of high and low market liquidity. In the model of high market liquidity, Net Operating Assets has a significant loading on the value factor and in the model of low market liquidity, there is also significant loading on the profitability and volatility factors, as well as the value factor. The O-Score portfolio has significant size and profitability betas in the high liquidity model and significant size, profitability and investment betas in the low liquidity model. Return on Assets, in the model of periods following high market liquidity, loads significantly on the size, value, profitability and momentum factors whereas in the model of periods following low market liquidity there is only significant loading on the size, profitability and momentum factors. Finally, for the Stock Issuance portfolio, there are significant size, profitability and investment betas in the model of high liquidity and significant market, profitability and investment betas in the model of low liquidity. 

Comparing the two liquidity six-factor plus volatility models with the unconditional six-factor plus volatility models, none of them has any anomaly portfolio with a significant alpha at the 5\% level of significance. The lack of significance means that the null hypothesis that the alpha equals zero cannot be rejected for any of the portfolios. Moving to the volatility factors in the three models, the Momentum, Net Operating Assets, O-Score and Return on Assets portfolios all significantly load on the volatility factor in the unconditional model. In the model of periods following high market liquidity none of the portfolios has significant volatility betas and in the model of periods following low market liquidity the only portfolio with significant loading on the volatility factor is the Net Operating Assets portfolio. 

\import{tables/}{liq-models}
\FloatBarrier

\section{Conclusion}
The introduction of the volatility factor into the unconditional six-factor model kills off any alpha generated in the anomaly portfolios. In both the conditional model of periods following high volatility and the conditional model of periods following high economic policy uncertainty, Stock Issuance generates significant alpha after the addition of the volatility factor. The O-Score portfolio survives the introduction of the volatility factor to the six Fama and French factors in the low volatility conditional model. Both the Accrual and Gross Profitability anomalies generate significant alpha in the low economic policy uncertainty model of the six Fama and French factors and the volatility factor. None of the anomaly portfolios generate significant alpha in the six-factor plus the volatility factor model for either of the high or low market liquidity conditional models. 

\subsection*{Acknowledgement}
I would like to acknowledge Beam Aschakulporn for his help and guidance with this research project.

\bibliography{master}
\bibliographystyle{apalike}
\end{document}                          % Document ends here
